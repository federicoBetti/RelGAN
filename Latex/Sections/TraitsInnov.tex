Innovation is a long process that requires a concomitance of circumstances and events to have a substantial impact in out society. However, the development of innovation is a researched process that in most of the cases has followed a common path, characterized by three main steps:
\begin{enumerate}
	\item[\textbf{1}] Generating novel and useful ideas
	\item[\textbf{2}] Championing these ideas to others
	\item[\textbf{3}] Implementing them toward better procedures, practices or products
\end{enumerate}
In order to achieve this goal, a person with a strong personality and well-defined characteristics is required: it will be called 'innovator'.

\paragraph{Personality and self-concepts} 
The analysis of human personality is a discipline born at the end of the 19\textsuperscript{th} century, but structured and better developed in the second half of the last century when, from a parallel research of different groups of researchers, a model called \textit{Five Factor Model} (FFM), or also 'The Big Five', was devised.\textsuperscript{\cite{BigFive5}} This model is composed of five orthogonal characteristics, useful to define a specific personality. \\
The features that characterize this model are presented below and can be summarized with the acronym OCEAN\textsuperscript{\cite{FFMDescription}}:
\begin{itemize}
	\item \textit{Openness to experience}: open-mindedness, willingness to try new activities, even not in accordance with your existing assumptions, curiosity about everything is new and unusual, including art and music
	\item \textit{Conscientiousness}: ability to complete tasks and to take into account the consequences of actions, organization, punctuality, set challenging goals and do everything possible to achieve them
	\item \textit{Extraversion}: being able to relate and speak to other people, feel pleasure and feel comfortable in team situations, not being intimidated by others
	\item \textit{Agreeableness}: altruism, ability to work in teams, seeking peace in dialogue, avoiding unpleasant discussions, enjoy relations with others
	\item \textit{Neuroticism}: emotional instability, high stress caused by over-thinking on problems, worrying only about negative aspects of situations
\end{itemize}
With these factors you can describe the personality of each type of person, but only a special blended mix of the five characteristics could make a good worker, engineer or scientist an innovator too. \\ The innovator should exhibit an high degree of Openness to Experience to develop the ability and drive the creativity in the direction to produce a new idea, but also a lot of Conscientiousness to have the strength to carry out difficult and suddenly complicated tasks that could cause a normal person to waver. Moreover, due to possible and sudden problems that may arise, Neuroticism has a fundamental value in the personality of an innovator; often an innovation, since it can be considered as a disruptive novelty, will find it difficult to be accepted by society and therefore reluctance and criticism could be fatal. In addition, teamwork plays a key role in the developing process an innovation: workers employed by an innovator must have an high potential to ensure the capability to follow the dream and to overcome difficulties. Only with a high level of Extraversion and Agreeableness an innovator can achieve it and create a close-knit team that fight for the same goal. Another key factor in the mentality of an innovator, essential for the success of a new project, is \textit{leadership}. A leader must be able to analyze each situation with courage, having the humility to ask for help but the resilience not to be influenced\textsuperscript{\cite{CoraggioHBR}}. As Bill Gates personally said when asked about innovators: "It's not an overnight type of thing; it's a kind of lifetime mindset"\textsuperscript{\cite{BGLifetimeMindset}}.

\paragraph{Cognitive Abilities, Knowledge and Motivation}
A proactive personality is not enough to become an innovator. Behind the figure of the innovator lies a great deal of general knowledge and especially in the specific domain. Most of the innovators we can admire today have a background made from a deep dedication and work ethic, passion and love for their innovation. However, this is not enough, it is important to be intelligent and smart, a characteristic that many show to be largely hereditary\textsuperscript{\cite{DEARY2013R673}}, to develop both challenging behavior in respect of others and an open mind that stimulates curiosity, motivation and thirst for new knowledge. Finally, another couple of factors are important: faith and luck. According to Virgil's dictum "Audentis fortuna iuvat".