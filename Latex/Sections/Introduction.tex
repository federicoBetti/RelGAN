(At least 2-3 pages of introduction) \\

Deep Learning techniques have proven in recent years to have great performances in various areas that can be connected with the real world. Clear examples of research papers and successful applications can be found in Computer Vision (cit cit cit), Robotics (\cite{Lillicrap} cit cit), Audio signal processing (cit), autonomous driving (\cite{ChenAutonomousDriving} \cite{AutonomousDriving2}), Medicine (\cite{Medicine} ) and, last but not least, Natural Language Processing (NLP) (cit cit). \\
Natural Language Generation (NLG) is a crucial topic inside the Natural Language Processing area and consists in the process of generating text starting from other sources. Now it is the center of major work that are developed by the biggest research centres in the world. In the past NLP tasks have been approached with classical grammars and ontology based methods. Now, after the deep learning revolution, the availability of big data in text, the capability of new GPU generations, NLP and the text generation research performed disruptive improvements. These new techniques have brought new life to the world of NLP research and allowed us to imagine increasingly innovative solutions with applications that in the future will be able to interact with humans through the use of language. 
It is therefore clear that the topic is of great relevance in scientific research with numerous conferences and journals where scientists from around the world can share the discoveries made. \\
The objective of the thesis is therefore to study these modern techniques of text generation in order to try to replicate them and try to propose an improvement in a specific direction. The world of Text Generation is made up of several parallel trends that use different technologies and methodologies, which have strengths and weaknesses. As can be seen from this fact, the world is still uncertain but in continuous evolution.

\subsection{Context}
NLG is the process of generating text starting from any type of data: starting from structured information in a database, starting from images or video frame or starting directly from other text. \\
This work is placed in the middle between these new Natural Language Processing approaches and brand new Deep Learning techniques. It is based on Deep Learning models called Generative Adversarial Networks. This architecture was first proposed in 2016 by a Montreal researcher called Ian Goodfellow \cite{GoodfellowGAN} and they have become firmly part of the scientific community that has immediately understood the potential. \\
These models, which will be analyzed in more detail in the work, have the ability to generate any distribution of data. This process is done thanks to models of complex neural networks that, based on real data, try to replicate these data with the greatest possible fidelity so that not even a human is able to distinguish the true samples from the generated ones. These models have shown incredible results in the field of Computer Vision and in recent years are becoming popular to model any type of data distribution, continue or discrete.

\subsection{Proposed Approach}
The thesis has multiple goals and objectives:
\begin{enumerate}
	\item Provide a extensive survey on modern deep learning models used for text generation and topic modelling
	\item Comparative analysis of different approaches, namely LSTM-based, Transformer-base with brute force (OpenAI) and new emerging GAN-Based methods
	\item Propose a new architecture for text generation, capable to generate short sentences guided by a topic and a given semantics, comparing it with very recently published papers on which it is based.
	\item Explore the features of the new GAN with multiple-discriminator that is useful for many different contexts. 
	\item Explore possible solutions and applications ( possibly in Italian to leverage on the capability of training the model from scratch each time). 
	\item Provide a system prototype on Twitter data.
	
\end{enumerate}

\subsection{Future Applications}

Although this topic is still studied a lot, it is clear which can be possible application in the future. This thesis could be considered as a small building block that goes directly to the direction of a future where these technologies, that now are just developed in highly innovative research centers, will be used commonly during daily-life activities. \\
Some of the possible applications that are based on Text Generation are: 
\begin{itemize}
	\item News Generation (newspapers and blog), conditional text generation wit a specific style
	\item Data-To-Text: create reports starting from structured data in databased or Excel sheet
	\item Retrieve information from text, audio, images to match content of different resources
	\item Man-Machine interaction for people with disabilities
	\item Dialogue Generation
	\item Machine Translation
	\item Text Summarization
	\item Image Captioning
\end{itemize}